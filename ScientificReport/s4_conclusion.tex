\chapter{Schlussfolgerungen}
\label{ch:conclusion}

Der Vergleich der genannten Verfahren hat gezeigt, dass jedes Klassifikationsverfahren seine spezifischen Vor- und Nachteile besitzt, um damit Spam-Nachrichten zu klassifizeren. Der kNN-Klassifkator liefert gute Ergebnisse, hat aber die längste Laufzeit aller drei verglichenen Verfahren. Ein großer Vorteil des Perceptron-Klassifiers ist, dass die Lernphase von der Klassifizerungsphase getrennt ist. Insgesamt liefert der Perceptron-Classifier bei im Vergleich zum kNN-Klassifikators reduzierter Laufzeit ähnlich gute Ergebnisse wie der kNN-Classifier, weist jedoch eine höhere Anzahl an false positives aus. Die Klassifizierung unter Verwendung der Mahalanobis-Distanz hat sich als das schnellste Verfahren herausgestellt, allerding zum Preis eines insgesamt schlechteren Klassifizierungs-Ergebnisses.

Wir konnten ebenfalls zeigen, dass sich mit einem Varianzen-basierten Verfahren die Anzahl der Features zu einem gewissen Grad reduziert werden kann, um dadurch Laufzeit einzusparen und dennoch akzeptable Ergebnisse zu erhalten. Ein Verfahren basierend auf iterativer Entfernung von Features mit geringem Einfluss hat sich hingegen als nicht zielführend erwiesen.